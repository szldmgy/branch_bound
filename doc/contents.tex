\section{Wstęp} % (fold)
  \label{sec:wstep}

  Celem pierwszej części laboratorium było zaimplementowanie i przetestowanie
  w praktyce algorytmu ,,Branch and Bound''. Algorytm ten umożliwia szybkie
  przeszukiwanie drzewa rozwiązań pomijając przeglądanie gałęzi, które na pewno
  nie zawierają rozwiązania optymalnego.

  Algorytm miał rozwiązywać jednoprocesorowy problem szeregowania zadań przy
  kryterium minimalizacji ważonej sumy opóźnień zadań.

  Problem ten można zdefiniować w następujący sposób

  Danych jest $n$ zadań, które mają być wykonane bez przerywania na pojedynczym
  procesorze, który może wykonywać najwyżej jedno zadanie jednocześnie.
  Każde zadanie $j$ jest dostępne do wykonania w chwili zero. Do wykonania
  wymagane jest $p_j$ jednostek czasu oraz zadanie ma określony priorytet $w_j$
  i określony czas zakończenia wykonywania $d_j$.
  Zadanie $j$ uznaje się za spóźnione, jeśli zakończy się wykonywać po terminie
  $d_j$. Miarą tego opóźnienia jest wielkość $T_j = \texttt{max}\{0, C_j - d_j\}$,
  gdzie $C_j$ jest terminem zakończenia wykonywania zadania $j$. Problem polega
  na znalezieniu takiej kolejności wykonywania zadań, aby zminimalizować
  kryterium $\sum w_j \cdot T_j$.

% section wstep (end)

\section{Implementacja} % (fold)
  \label{sec:impl}

  Algorytm ,,Branch and Bound'' zaimplementowałem w języku C. Permutacje są
  generowane za pomocą algorytmu rekurencyjnego. Rozwiązanie początkowe jest
  generowane losowo. Główna część algorytmu realizowane jest przez funkcję
  \texttt{permute}, natomiast funkcja \texttt{compute} przygotowuje bufory,
  które wykorzystywane będą przez funkcję \texttt{permute}.

  Lista zadań czytana jest ze standardowego strumienia wejściowego. Rozwiązanie
  to zostało użyte w celu uproszczenia programu. Zarówno na systemie Windows
  oraz *NIX możliwe jest przekierowanie zawartości pliku do standardowego
  strumienia wejściowego za pomocą operatora \texttt{<} podczas wywoływania
  programu.

  Zaimplementowane zostały następujące procedury eliminacyjne.
  \begin{enumerate}
    \item Jeśli dolne ograniczenie dla częściowej permutacji jest większe niż
          górne ograniczenie, gałąź ta zostaje odrzucona;
    \item Jeśli po przeniesieniu ostatniego zadania na pozycje $x = 1,\ldots,k-1$
          i obliczeniu $L_x = \sum_{j=1}^{k} w_j \cdot T_j$. Jeśli $\exists x, L_x < L_k$,
          gałąź ta zostaje odrzucona.
  \end{enumerate}

  Jeśli jedna z wcześniej uruchomionych metod odrzuci gałąź, kolejne metody nie
  są uruchamiane.

% section impl (end)

\section{Wyniki} % (fold)
  \label{sec:wyniki}

  Program został uruchomiony dla zadanych instancji w trzech wersjach.
  Pierwsza z nich nie korzystała z żadnej procedury eliminacyjnej (przegląd
  zupełny), druga korzystała z jednej metody eliminacji (porównanie dolnego
  ograniczenia z górnym), a trzecia ze wszystkich zaimplementowanych metody
  eliminacyjnych.

  Wyniki działania programów znajdują się w poniższej tabeli.

  \begin{center}
    \begin{tabular}{c}

    \end{tabular}
  \end{center}

  Widać jednoznacznie, że większa liczba metod eliminacyjnych powoduje znaczne
  skrócenie czasu przeszukiwania drzewa rozwiązań.

  Testy zostały przeprowadzone na procesorze Core 2 Duo, a program skompilowany
  został za pomocą kompilatora Clang\footnote{http://clang.llvm.org}. Program
  zawarty na płycie CD skompilowany został kompilatorem GCC i zawiera wszystkie
  metody eliminacyjne.

% section wyniki (end)

\section{Wnioski} % (fold)
  \label{sec:wnioski}

  Algorytm jest bardzo skuteczny dzięki zastosowaniu metod eliminacyjnych, które
  powodują znaczne skrócenie czasu przeszukiwania drzewa rozwiązań.
  Czas odnajdywania rozwiązania można znacznie przyspieszyć stosując algorytm
  aproksymacyjny, bądź heurystyczny, generowania rozwiązania początkowego.
  Algorytm taki nie został zaimplementowany, gdyż nie był to cel tego zadania.

% section wnioski (end)

\newpage
\appendix
\section{Pełen listing kodu} % (fold)
  \label{sec:listing}
  \singlespacing
  \inputminted{c}{../bb.c}

% section listing} (end)
